\chapter{Introduction}

One of the most well-known, well-studied and well-understood algorithmic problems is to find the \textsc{Shortest Path} in a graph. The problem is simple: given a graph and two vertices, find the shortest sequence of edges to go from one vertex to the other. Yet, the applications are almost limitless: to find the fastest route home, to find the cheapest airline tickets to Kuala Lumpur, to solve a Rubik's Cube in the fewest moves, to determine the best-case running time of an algorithm, or to move an enemy in a video game.

This thesis, however, is about a curious little variant, called the \textsc{Shortest Odd Path}. Now we consider only paths consisting of an odd number of edges. If you were to step through the graph, and start walking with your right foot, then an odd path is one where you would also end up on your right foot. The applications of this variant are not remotely as obvious. It rarely, if ever, matters whether a path has odd or even length in any of the examples mentioned above, and it is difficult to come up with example problems where it does matter.

The reason we care is because many other more useful problems are much easier to solve if we already have an algorithm to determine the shortest odd path in a graph. Consider for example the problem of the \textsc{Shortest Bottleneck Path}: find the shortest path from one vertex to another, except that we are also given a 'bottleneck' edge with the requirement that the path has to go through the bottleneck. Imagine doing a road-trip in Norway, but for the complete road-trip experience you also really want to drive through Norway's longest tunnel, and preferably without using the same roads more than once. Coming up with an algorithm for this is not as simple as it sounds, but we will show that it is much easier if we know how to solve \textsc{Shortest Odd Path}.

An even more directly useful problem to solve is the one of \textsc{Network Diversion}: given two vertices and a marked edge in a graph, find the cheapest set of edges to delete from the graph such that all paths from one vertex to another must pass through the marked edge. This one has more immediate practical applications: imagine you are a military commander in a war, you know that the enemy wants to move their troops and supplies, and you are very prepared to ambush them if they cross a certain bridge. Now what is the fastest or cheapest way to destroy bridges to funnel the enemy through the ambush?

Solving \textsc{Network Diversion} efficiently is no simple task, and many of its variants have been proved to be NP-complete. Whether there exists an algorithm to solve \textsc{Network Diversion} in polynomial time on undirected \emph{planar graphs} is for now an open problem, and it turns out the answer is yes: it is possible if you already have an efficient algorithm to solve \textsc{Shortest Odd Path}. This is the topic of our thesis. We develop and implement an efficient algorithm to solve \textsc{Shortest Odd Path}, and then use that to implement the first-ever efficient algorithm for \textsc{Network Diversion} on planar graphs.

\section*{Overview of the contents}
We start this thesis with some preliminaries, mainly around graph theory, in \Cref{chapter:preliminaries}. Then we warm up our problem-solving skills in \Cref{chapter:odd-walk}, where we solve a much easier variant of \textsc{Shortest Odd Path}, called \textsc{Shortest Odd Walk}. In \Cref{chapter:odd-path} we reach the star of the thesis: our algorithm for \textsc{Shortest Odd Path}. With this star in hand we can head to \Cref{chapter:network-diversion} to solve \textsc{Network Diversion} for planar graphs. 

Most of the algorithms discussed here have also been implemented and tested in practice \cite{source:codebase}, and \Cref{chapter:codebase} presents the codebase. In addition, the source code for this thesis itself can be found at \cite{source:thesis}.

The reader may visit the chapters and source code in any order they like, though we would like to suggest a chronological order if nothing else.
