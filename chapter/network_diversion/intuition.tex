\section{Intuition}
\subsection{Bottleneck Paths}
Before we reveal the algorithm for Network Diversion, we will first look at a curious little problem that we call Shortest Bottleneck Path.

\fbox{\parbox{0.94\textwidth}{\textsc{SBP: Shortest Bottleneck Path}\\
    \textbf{Input:} A graph $G$, two vertices $s,t \in V$, and a 'bottleneck' edge $b \in E$\\
    \textbf{Output:} the shortest $s$-$t$-path in $G$ that goes through the bottleneck $b$
}}

There is no obvious way to solve SBP. One might attempt to find the shortest paths from $s$ to $from(b)$ and from $to(b)$ to $t$, but those two paths might overlap and reuse the same vertices, and therefore would their concatenation not necessarily be a simple path.

Instead we create a new graph $H$, by subdividing all edges in $G$ \emph{except} $b$, like seen in figure TODO. The key point to see here is that any odd $s$-$t$-path in $H$ must necessarily go through the bottleneck, otherwise it would not be odd. We can visualize it by 'stepping through' the edges in $H$. If we start on our right leg, then in the beginning every time we reach a vertex that is also in $G$, we reach it by stepping on our left leg. That continues until we use the bottleneck edge, and from then on we step on all vertices from $G$ using our right leg. If we require that we must end at $t$ on our right leg, then the path must be odd, and any odd path must go through the bottleneck. Therefore we can simply run our Shortest Odd Path algorithm on $H$, and if such a path exists we can reverse the subdivision of the edges in the path and the result is the Shortest Bottleneck Path in $G$.

\subsection{Extending the idea to multiple edges}
If we extend the problem to have multiple bottleneck edges, and we have to go through all of them, then our idea will not work. That is good, because otherwise we would have solved the Traveling Salesman Problem in polynomial time and complexity theory as we know it would break down. TODO fact check.
The problem is that we have no way of knowing whether we have used the marked edges 1, or 3, or 5, etc. times, because in all of them we hit vertices from $G$ using our right leg. We can, however, use this idea to find paths that use a certain set of edges an odd amount of times. And as it turns out, that is exactly what we need to solve Network Diversion.

TODO: Intuition for ND