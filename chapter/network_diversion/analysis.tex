\section{Analysis}

\begin{theorem}
    Let $(G, s, t, d)$ be an instance of \textsc{Network Diversion}, and let $n~:= |V|$. \\
    Claim: our algorithm runs in time $O(n \log n)$.
    
    \begin{proof}
        We find first a shortest $s$-$t$-path in $G$ that does not use $d$, in time $O(n+m)$.
    
        Then we subdivide all the edges in $G^\star$ except those found in the path, in time $O(n+m)$. This new graph has size $n' \leq 2n \in O(n)$ and $m' \leq 2m \in O(m)$.
    
        Next up is to find an odd path in the subdivided graph, in time $O(m' \log n') = O(m \log n)$.
    
        Lastly, if we are interested in the specific set of edges in the diversion and not just the cost, we un-subdivide the odd path in time $O(n') = O(n)$.
    
        In total, we have a running time of $O(n+m) + O(m \log n) + O(n) = O(m \log n)$. Since $G$ is planar we have that $m \in O(n)$, and we can simplify the complexity to just $O(n \log n)$, which completes the proof.
    \end{proof}
\end{theorem}

Note that here we have assumed that the dual graph $G^\star$ has already been computed prior to starting the algorithm. If we have a straight-line embedding of $G$ we can compute $G^\star$ in $O(n+m)$, which would not change the overall running time. However, if we do not have such an embedding the total running time might be considerably more.

We compare the theoretical and practical running times on Delaunay graphs as we did in \Cref{subsubsection:odd-path-delaunay-testing}. For each graph, we have estimated a source and target vertex of maximum distance between each other and picked three edges in the graph as diversion edges. We pick whichever diversion edge leads to the worst running time over 10 runs, and plot the median over those 10 runs.

Here too have we tried to create a function out of the theoretical running time of $O(n \log n)$, this time with different constants. We have set $m~:= 3n$ in the plot since the graphs are planar.

\begin{center}
    \includesvg[width=0.9\textwidth]{figures/bench_plots/network diversion.svg}    
\end{center}

As we can see, the running times grow just barely more than linearly compared to the input size. This is not surprising considering the linearithmic theoretical running time. The algorithm easily solves \textsc{Network Diversion} on planar graphs of 100000 vertices in less than a second. Now compare that to the existing algorithms that will need more than a second to solve for more than 30 vertices, and it is clear why a polynomial running time matters so much.

See \Cref{figure:network-diversion-on-delaunay} for yet another example of what a diversion set may look like, this time on a Delaunay graph of 35 vertices.

\noindent
\begin{figure}
    \centering
    \begin{subfigure}{.32\textwidth}
        \centering
        \includesvg[width=\textwidth]{figures/diversions/delaunay35-Full.svg}
        \caption{$s$ and $t$ are marked in blue, the diversion edge in red}
        \label{subfigure:network-diversion-input}
    \end{subfigure}\hfill%
    \begin{subfigure}{.32\textwidth}
        \centering
        \includesvg[width=\textwidth]{figures/diversions/delaunay35-Partial.svg}
        \caption{One possible diversion set marked in orange}
        \label{subfigure:network-diversion-diversion}
    \end{subfigure}\hfill%
    \begin{subfigure}{.32\textwidth}
        \centering
        \includesvg[width=\textwidth]{figures/diversions/delaunay35-Diverted.svg}
        \caption{With the diversion set removed, all $s$-$t$-paths must go through the diversion edge}
        \label{subfigure:network-diversion-diverted}
    \end{subfigure}\hfill%
    \caption{Example of a solution for \textsc{Network Diversion}}
    \label{figure:network-diversion-on-delaunay}
\end{figure}
