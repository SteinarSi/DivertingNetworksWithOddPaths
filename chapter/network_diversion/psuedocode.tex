\section{Psuedocode}
Here comes the psuedocode for our \textsc{Network Diversion} algorithm. For a run-through of the algorithm on an example graph, see \Cref{subsection:network-diversion-example}.
% /**
%     In:
%         graph: an undirected planar graph, with non-negative edges 
%         s, t: vertices whose $s$-$t$-paths we consider
%         d: the diversion edge

%     Out:
%         The cheapest set of edges that can be deleted from the graph 
%         to force all $s$-$t$-paths to go through $d$, along with the 
%         cost, if one such set exists.
% */
\begin{lstlisting}[caption={Main},label=Listing,mathescape=true]
fn network_diversion(graph, s, t, d) {
    graph.delete_edge(d);
    path = shortest_path(graph, s, t);
    graph.add_edge(d);

    match path {
        None => {
            // No $s$-$t$-paths exist without $d$ anyway,
            //  so no diversion is needed.
            return Some(0, []);
        }
        Some(p) {
            p$^\star$ = [ e$^\star$ for e in p ];
            dual = subdivide_edges_except(graph$^\star$, p$^\star$);

            match shortest_odd_path(dual, left(d), right(d)) {
                // There are no $s$-$t$-paths that go through $d$,
                //  and therefore no way to divert the network.
                None => return None;
                Some(cost, odd_path) {
                    diversion = [e for e$^\star$ in un_subdivide_edges(odd_path)];
                    return Some(cost, diversion);
                }
            }
        }
    }
}
\end{lstlisting}
