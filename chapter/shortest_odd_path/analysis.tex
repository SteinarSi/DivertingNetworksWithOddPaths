\section{Analysis}

\subsection{Complexity}
Let $(G,s,t)$ be an instance of \textsc{Shortest Odd Path}, let $n := |V|$ and let $m := |E|$. \\
Claim: the algorithm runs in time at most $O(m \cdot \log n)$, or $O(m \cdot \log n)$ of the graph is simple.
\begin{proof}  
    First of all, we construct the mirror graph $H$ with $2n-2 \in O(n)$ vertices and $2m - deg(s) - deg(t) \in O(m)$ edges, in time $O(n+m)$.
    
    With our \pyth{completed} array we can guarantee that each vertex is scanned at most once, and the scanning operation just loops through all the neighbours. Therefore, the total cost of all the scans is $O(n + \sum_{u \in V}) deg(u) = O(n + 2m) = O(n + m)$.

    The blossom operation is a little more complicated. Thanks to the overly complicated code in our \pyth{backtrack_blossom} procedure, we can backtrack from a blossom edge and determine the vertices in the blossom in time linear to the size of the blossom. Setting their values for $d^+$ and $d^-$ can also be done in linear time, and the potential scans have already been accounted for above. The key point here is that we shrink the blossom into a psuedonode afterwards: each vertex can then only be part of such a blossom procedure at most once. Even though we may compute many blossom edges, the total amount of work will therefore never exceed $O(n)$.

    Lastly, we have the main loop, which iteratively pop vertices and blossom edges from the queue. Each of the $O(n)$ vertices may be put into the priority queue many times, at most once for each of its neighbours. That is a total of $O(m)$ vertices in the queue, for a total cost of $O(m)$. Though it is unlikely, in the extreme case all edges may be put in the queue as blossom edges as well, again for a total cost of $O(m)$. In total, enqueueing everything costs $O(m)$, and dequeueing everything costs $O(m \cdot \log m)$.

    In total, the algorithm runs in time $O(n+m) + O(n+m) + O(n) + O(m) + O(m \cdot \log m) = O(m \cdot \log m)$.
    If the graph is simple, then we can simplify it further to $O(m \cdot \log n^2) = O(m \cdot 2 \cdot \log n) = O(m \cdot \log n)$.
\end{proof}

\subsection{Benchmarking methodology}

\subsection{Results}
\subsubsection{Testing different data structures for the Basis}
\label{testing-basis}
As shown and discussed in \ref{basis-code}, we have implemented multiple data structures to keep track of the basis of each vertex. One of them is based on the Observer pattern, the other on the well-known UnionFind structure. We have tested both on 6 different graphs, and we show the results below. We ran our shortest odd path algorithm on each graph 40 times,20 for each structure, and noted the fastest times for each. The vertices with id's 0 and n-1 were chosen as the source and target to paths for, respectively. \todo{reformat}
\begin{center}
    \begin{tabular}{|l | r | r | r | r | c|} 
     \hline
     Graph & n & m & Observer & UF & Change \\ [0.5ex] 
     \hline\hline
     Oldenburg & 6105 & 7035 & 5.1555 ms & 4.6955 ms & -11.277\%\\ 
     \hline
     San Joaquin County & 18263 & 23874 & 6.7046 ms & 5.7347 ms & -14.212\%\\
     \hline
     Cali Road Network & 21048 & 21693 & 19.280 ms & 19.119 ms & +0.1590\%\\
     \hline
     Musae Github \cite{musae-github} & 37700 & 289003 & 21.375 ms & 19.438 ms & -2.4708\%\\
     \hline
     SF Road Network & 174956 & 223001 & 93.494 ms & 87.570 ms & -6.3364\%\\ [1ex] 
     \hline
     Pokec Social Network \cite{soc-pokec} & 1632804 & 30622565 & 13.225 s & 13.774 s & +4.1543\%\\ [1ex] 
     \hline
    \end{tabular}
\end{center}
\todo{cite the rest of the graphs}
The results are a little mixed. The UnionFind-based structure spends $4.2\%$ more time on the huge Pokec Social Network graph. However, it also either outperforms or does as good as the Observer-based structure on all the other graphs, at best shaving off $14.2\%$ of the time spent on the San Joaquin County graph. Since the UnionFind-based structure outperforms the Observer-based structure on average, we have chosen to use that one for the remainder of this thesis.

\subsection{Discussion}