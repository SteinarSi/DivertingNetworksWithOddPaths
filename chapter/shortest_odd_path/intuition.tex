\section{Intuition}
Now that we have tried out some algorithms for \textsc{Shortest Odd Walk}, we are finally ready to add the restriction that each vertex is used at most once, and thus solve \textsc{Shortest Odd Path}. The algorithm we are about to present is based on Ulrich Derigs' algorithm from 1985 \cite{derigs_shortest_odd_path}, though with some improvements.

\subsection{Reduction to \textsc{Shortest Alternating Path}}
Consider first another related problem:

\fbox{\parbox{0.94\textwidth}{\textsc{Shortest Alternating Path}\\
    \textbf{Input:} A weighted graph $G := (V, E)$, two vertices $s,t \in V$, and a set $F \subseteq E$\\
    \textbf{Output:} the shortest $s$-$t$-path in $G$ where every other edge used is in $F$.
}}

Derig observed that \textsc{Shortest Odd Path} can be reduced to a special case of \textsc{Shortest Alternating Path}, by constructing what we will refer to as a $\emph{mirror graph}$.

\begin{definition}[Mirror graph]
    Let $G = (V, E)$ be a graph, and $s,t \in V$ be two vertices.
    We construct a new graph $H \sqsupset G$, where for each vertex $u \in V \setminus \{s,t\}$ we add a 'mirror' vertex $u'$, and a connecting 'mirror' edge between them. 
    The vertices in $V(H)$ that are also in $V(G)$ are referred to as the 'real' vertices, and the newly added vertices are referred to as the 'mirror' vertices. In addition, for any vertex $u \in V(H) \setminus \{s,t\}$, real or not, we define $mirror(u)$ as $u$'s mirror on the other side. We usually label mirror vertices with an $'$ at the end of the real counterpart's label.
    For example, if $G$ is the graph in Figure ~\ref{small5}, then Figure ~\ref{small5-2} would be its corresponding mirror graph $H$.
\end{definition}

Our reduction from \textsc{Shortest Odd Path} to \textsc{Shortest Alternating Path} follows:
\begin{enumerate}
    \item Let $(G, s, t)$ be an instance of \textsc{Shortest Odd Path}.
    \item Construct $H$ as the mirror graph of $G$, and let $F$ be the set of mirror edges in $H$. Now $(H, s, t, F)$ is an instance of \textsc{Shortest Alternating Path}.
    \item Let $P'$ be the shortest alternating path of $(H, s, t, F)$, if one exists. If none exist, then we do not have any odd $s$-$t$-paths in $G$ either.
    \item Construct $P$ by filtering out mirror edges from $P'$, and for each edge $(u',v') \in E(H) \setminus (F \cup E(G))$ from the mirror side of $H$ we replace it by the corresponding edge $(u,v) \in E(G)$ from the real side.
    \item Now $P$ is the shortest odd $s$-$t$-path in $G$.
\end{enumerate}

\begin{figure}
    \centering
    \includesvg[width=15cm]{figures/graphs/small5.svg}
    \caption{Our input graph $G$, for \textsc{Shortest Odd Path}}
    \label{small5}
\end{figure}

\begin{figure}
    \centering
    \includesvg[width=15cm]{figures/graphs/small5-2.svg}
    \caption{The mirror graph $H$ of $G$, with mirror edges marked in red and mirror vertices labeled with an $'$}
    \label{small5-2}
\end{figure}

For example, if our input $G$ for \textsc{Shortest Odd Path} is Figure ~\ref{small5}, then $H$ and $F$ could look like Figure ~\ref{small5-2}. One of the two possible alternating paths is $P' := [(s,a), (a, a'), (a',b'), (b',b)$, $(b,c), (c,c'), (c',d'), (d',d), (d,t)]$. When we filter out mirror edges and replace edges from the mirror side with their real counterparts, we end up with $P := [(s,a),(a,b),(b,c),(c,d),(d,t)]$, which is one of the two possible odd paths of $G$. 

TODO dette kan kanskje formuleres bedre?

To see why the reduction works, simply observe that for each step we take in the graph, we have to go to the other side of the mirror. If we take another step, we get back to the same side again. It is only when we reach the target vertex $t$ that we do not have to go to the other side. Therefore, to reach a neighbour of $t$, we must have used an even number of mirror edges and an even number of non-mirror edges, and when we take the last step to reach $t$ we have used an odd number of edges and thus found an odd path. If this alternating $s$-$t$-path in $H$ is the shortest such path, then the corresponding path in $G$ must also be the shortest odd $s$-$t$-path in $G$. The reduction works also in the weighted case, as long as each edge $(u',v')$ on the mirror side get the same weight as their real counterpart, and all the mirror edges get the same (usually 0) weight. The interested reader may see \cite{derigs_shortest_odd_path} for more details on this reduction.

\subsection{Adapting previous work}
Ball and Derigs \cite{shortest_alternating_path} have shown how to efficiently solve \textsc{Shortest Alternating Path}. In their algorithms, subgraphs are shrunk into psuedonodes whenever possible, to make the graph smaller. The drawback is that certain psuedonodes must later be expanded again, which is the most complicated and expensive part of their algorithms. In our case, however, we have a special case of \textsc{Shortest Alternating Path}. The set $F$ is, with the exception of $s$ and $t$, a perfect matching of $H$, and we will therefore never have to expand psuedonodes after shrinking them. The curious reader may visit \cite{shortest_alternating_path} for more on these algorithms and why our almost-perfect matching is a simpler case.

TODO siter \cite{blossom} her et sted



