\section{Data structures}
To perform these algorithms, we have implemented a few different data structures. The first is a basic structure for undirected graphs, using a vector of vectors of edges to represent the graph with adjacency lists. It is generic in the type of its edge weights, but also in the type of its edges. The 'basic' edge has the basic edge methods like \pyth{.to()} and \pyth{.weight()}, but we also have a struct for planar edges, where we have the methods \pyth{.left()} and \pyth{.right()}. The methods work as described in \Cref{section:graphs} and \Cref{section:planar-graphs}. 

We also have a data structure for planar graphs. They consist of two undirected graphs of planar edges: the 'real' graph and its dual. Each of them has edges that know which faces are to their left and right in the other graph. Even though our algorithm for \textsc{Network Diversion} will work with any planar graphs, parsing them and finding an embedding is rough. We therefore assume that we are the given coordinates of each of the vertices and that they form a straight-line embedding. From there we can compute the dual. Whether the given coordinates form true planar embeddings is usually not checked, since the code to verify that runs in the painfully slow $O(m^2)$ and has been disabled by default.

Another limitation is that we can only handle simple planar graphs: if we have parallel edges, then the way we compute the dual will not work. If the input graph is not simple, we have to somehow combine the parallel edges until it is. Depending on the use case we have many reasonable strategies for combining them. If the edges represent bridges to be blown up with artillery, then the combined edge should probably be the sum of the weights of its components, the sum of the artillery rounds needed to destroy the edges between those two vertices. If we in another case just need to cut any of the edges between them, then we may want to just take the cheapest one, or perhaps we are forced to take the most expensive one. Rather than making too many assumptions about the use cases, we provide five strategies for combining parallel edges, to cover as many use cases as possible: 

\begin{itemize}
    \item Take the first edge
    \item Take the last edge
    \item Keep the highest weight
    \item Keep the lowest weight
    \item Sum all weights
\end{itemize}

We hope that this is enough. If not, then our framework can easily be extended with more strategies.

As discussed in \Cref{subsection:basis-code} and benchmarked in \Cref{subsubsection:testing-basis}, we provide two data structures to keep track of the basis. They are called ObserverBase and UnionFindBase. Both are implemented using a common trait and can be switched out interchangeably in the \textsc{Shortest Odd Path} algorithm. UnionFindBase usually performs the best on sparse graphs, and has been set as the default, but ObserverBase may be preferable in denser graphs. The 'näive' basis we mentioned was too inefficient for anything but a temporary prototype and has long since been deleted.
