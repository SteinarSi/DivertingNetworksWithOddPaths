The crux of this thesis is not about the theory we have presented in the other chapters, but rather to see how well these algorithms work in practice, if at all. We have implemented most of the algorithms mentioned in the paper, and in this chapter we will present the codebase. Of course, we invite the reader to explore the repository on their own: \cite{source:codebase}.

\section{Functionality}
Our library is written in Rust. We chose that language because of performance, but also because its type system and strict compiler helps massively in reducing the developer overhead during development. Being able to fearlessly perform large refactors, and to use sum-types rather than just product types to describe the data, has had a huge impact on the quality of our code.

With that, we have successfully implemented the following algorithms:

\begin{itemize}
    \item \textsc{Shortest Odd Walk} on graphs of non-negative weights, as described in \Cref{chapter:odd-walk}.
    \item \textsc{Shortest Odd Path} on undirected graphs of non-negative weights, as described in \Cref{chapter:odd-path}.
    \item \textsc{Network Diversion} on undirected planar graphs of non-negative weights, as described in \Cref{chapter:network-diversion}.
    \item \textsc{Shortest Path} on unweighted graphs, using a classic breadth-first search. It also runs on weighted graphs, through happily ignoring the weights.
    \item \textsc{Shortest Path} on graphs of non-negative weights, using Dijkstra's Algorithm as shown in \Cref{algorithm:dijkstras-algorithm}.
    \item \textsc{Shortest Detour Path} on undirected graphs of non-negative weights, as described in \Cref{section:subdividing-detours}.
\end{itemize}

All of them are generic with respect to edge weights, and can handle any type of number-like weights such as \pyth{i32}, \pyth{u64} and \pyth{f64}. When we subdivide edges in \textsc{Network Diversion} and \textsc{Shortest Detour Path} we use a neat little trick to be able to split a weight into two weights without potentially accidentally rounding down an integer: we set one weight to zero and the other to the original weight. Any path that uses one of those edges will have to use the other afterwards, and their sum will then be the same as the original weight prior to the split.
