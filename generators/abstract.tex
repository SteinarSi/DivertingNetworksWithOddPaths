\pagenumbering{roman}

\begin{abstract} 
    The problem of \textsc{Shortest Odd Path} is to find a path from one vertex to another in a graph, where the number of edges in the path must be odd. Although the problem might seem like merely a curiosity, its utility is that many more useful problems are easier to solve if we know how to solve \textsc{Shortest Odd Path}.\\[3mm]
    One of these problems is called \textsc{Network Diversion}: given a graph, two vertices and a marked edge, compute the cheapest set of edges to delete from the graph such that all paths from one vertex to another must pass through the marked edge. Many of its variants are NP-complete, but its complexity on planar graphs remains an open problem. \\[3mm]
    We implement an efficient algorithm based on \cite{source:derigs_shortest_odd_path} to solve \textsc{Shortest Odd Path} on undirected graphs, and use that to implement the first-ever efficient algorithm for \textsc{Network Diversion} on planar graphs.

% \noindent Lorem ipsum dolor sit amet, his veri singulis necessitatibus ad. Nec insolens periculis ex. Te pro purto eros error, nec alia graeci placerat cu. Hinc volutpat similique no qui, ad labitur mentitum democritum sea. Sale inimicus te eum.

% No eros nemore impedit his, per at salutandi eloquentiam, ea semper euismod meliore sea. Mutat scaevola cotidieque cu mel. Eum an convenire tractatos, ei duo nulla molestie, quis hendrerit et vix. In aliquam intellegam philosophia sea. At quo bonorum adipisci. Eros labitur deleniti ius in, sonet congue ius at, pro suas meis habeo no.

\end{abstract}

% \renewcommand{\abstractname}{Acknowledgements}
% \begin{abstract}
% 	Est suavitate gubergren referrentur an, ex mea dolor eloquentiam, novum ludus suscipit in nec. Ea mea essent prompta constituam, has ut novum prodesset vulputate. Ad noster electram pri, nec sint accusamus dissentias at. Est ad laoreet fierent invidunt, ut per assueverit conclusionemque. An electram efficiendi mea.
% 	\vspace{1cm}
% 	\hspace*{\fill}\texttt{Your name}\\ 
% 	\hspace*{\fill}\today
% \end{abstract}

\setcounter{page}{1}
\newpage